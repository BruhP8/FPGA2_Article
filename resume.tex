\documentclass[10pt,a4paper]{article}
\usepackage[utf8]{inputenc}
\usepackage[french]{babel}
\usepackage[T1]{fontenc}
\usepackage{amsmath}
\usepackage{amsfonts}
\usepackage{amssymb}
\usepackage{graphicx}
\usepackage{hyperref}
\usepackage[left=2cm,right=2cm,top=2cm,bottom=2cm]{geometry}
\author{Legoueix Nicolas n°28709604}
\title{FPGA2 - Analyse documentaire et critique \\ Extending RISC-V for HW acceleration of Post-Quantum Scheme LAC \\ DOI 10.23919/DATE48585.2020.9116567}
\begin{document}
\maketitle
\newpage
\part{Résumé}

\section{Contexte}
\subsection{Necessité de nouveaux protocoles, travail effectué}
La majorité des communications actuelles sont réalisées à l'aide de l'algorithme de cryptographie RSA. Cet algorithme de cryptographie asymétrique utilise deux clés pour le 
chiffrement des données : une clé publique -crée par un des interlocuteurs et rendu publique- servant à chiffrer les données et une clé privée -connue seulement de l'interlocuteur devant déchiffrer 
les données et indispensable à ce processus-. Cette méthode se base sur le fait qu'il est impossible de déchiffrer un message sans la clé privée en un temps dit raisonable. En effet, le temps nécessaire 
pour calculer des factorielles (sur lesquels se base RSA) étant exponentiel, il est de plus en plus difficile de déchiffrer RSA par attaques de force brute.\footnote{Actuellement, le record est RSA-250 soit 829bits, 
établit en février 2020 par F. Boudot, P. Gaudry, A. Guillevic, N. Heninger, E. Thomé et P. Zimmermann. Les clés privées actuelles font actuellement entre 1024 et 2048 bits. Temps requis : 
2700 années CPU, c'est a dire qu'il aurait fallu 2700 années de travail à un seul coeur (dans ce cas, un Intel Xeon Gold 6130 a 2.1Ghz) pour résoudre le problème. Voir \href{https://eprint.iacr.org/2020/697.pdf}{le papier de recherche publié : DOI 10.1007/978-3-030-56880-1\_3} }.\\
Cependant, l'avènement des processeurs quantiques, et l'arrivée prochaine de vrais ordinateurs quantiques remet en question la résistance de l'algorithme RSA. En effet, ces derniers seraient théoriquement
capables de déchiffrer une communication RSA en seulement quelques heures de travail.\footnote{Voir \href{https://quantum-journal.org/papers/q-2021-04-15-433/pdf/}{DOI 10.22331/q-2021-04-15-433}}. Il devient donc nécessaire
de développer des algorithmes résistants aux attaques quantiques, ainsi que de créer le matériel qui sera capable d'exécuter ces nouveaux algorithmes. Parmis eux figurent l'algorithme LAC qui es présenté dans l'article proposé.\\~\\
Cet article propose une implémentation accélérée matériellement de l'algorithme LAC. Les auteurs y décrivent un total de quatre accélérateurs métériels visant à mitiger les effets des principaux goulots d'étranglements 
de LAC, tous intégrés dans le pipeline d'un coeur RISC-V modifié. Les auteurs décrivent également les modifications apportées à cette ISA afin de pouvoir utiliser leur matériel.

\subsection{Algorithme LAC}
L'algorithme se déroule en trois étapes :
\begin{itemize}
    \item Génération des clés : Un premier interlocuteur génère une clé publique et une clé privée. Cette génération est effectuée à partir d'une graine qui est ensuite étandue via l'algorithme
SHA256. Une multiplication polynomiale afin de créer une instance RLWE (Ring Learning With Error).
    \item Chiffrement : Le message à envoyer est chiffré via la clé publique à l'aide d'un encodeur BCH, puis transformé en un nombre polynomial. Enfin, ce polynome est "camouflé" dans l'instance 
RLWE qui introduit de polynomes parasites dans le message chiffré. Le message est ensuite envoyé.
    \item Déchiffrement : L'instance RLWE est reçue par le second interlocuteur en possession de la clé privée. On soustrait ensuite de l'instance le plus gros des polynomes parasites, puis on 
utilise un décodeur BCH (et la clé privée) pour soustraire le bruit restant. On obtient alors le message déchiffré.
\end{itemize}
Il est important de remarquer trois opérations utilisées dans cet algorithme : la génération de polynomes lors de la génération des clés, la multiplication de polynomes (lors de la création d'instance 
RLWE et lors de la suppression des polynomes parasites pendant le déchiffrement), et la correction d'erreurs.\\
La multiplication de polynomes est particulièrement complexe, il serait donc intéressant d'accélérer cette opération avec du matériel.

\section{Implémentations}
\subsection{Accélération matérielle}
Quatre accélérateurs sont décrits dans cet article, chacun visant à diminuer les effets des goulots d'étranglements inhérents à l'algorithme LAC.

\subsubsection{MUL-TER : Multiplicateur Polynomial Terniaire}
LAC se base sur la multiplication de nombre polynomiaux pour le chiffrement et le traitement du bruit lors du déchiffrement. Etant donné qu'il que beaucoup de multiplications sont effectuées 
sur ce type de nombres et que ce sont des opérations complexes, il est naturel de vouloir acclèrer leur traitement. Dans le cas présent, les auteurs sont partis d'un multiplicateur existant et y ont 
ajouté la capacité à effectuer des calculs de convolutions négatives. \\
MULTER est un banc de MAU (Modular Arithmetic Units) capables d'effectuer des additions, soustractions ou une opération neutre (forwarding) afin de réaliser une convolution bouclée à partir de deux polynomes.

\subsubsection{MUL-CHIEN}
L'algorithme LAC introduit des erreurs lors du (dé)chiffrement. Il est donc nécessaire de savoir détecter et corriger ces dernière efficacement. Le module MUL-CHIEN implémente en matériel 
l'algorithme de recherche de Chien afin de rechercher la racine des erreurs détectées. Cette opération nécessite de multiplier les éléments d'un polynome par une constante. Pour plus de flexibilité et 
de simplicité, les auteurs ont développé leur propre multiplicateur pour tenir ce role : MUL-GF (Gallois Field Multiplier) capable d'effectuer quatre opérations en parallèle en seulement neufs cycles à partir du travail 
d'autres chercheurs. MUL-CHIEN est composé d'un banc de MUL-GF.

\subsubsection{SHA256}
La génération des nombre polynomiaux, vitale à l'algorithme LAC, est faite par un accélérateur développé par les mêmes chercheurs que ceux écrivant cet article. 
\footnote{Voir DOI:10.1007/978-3-030-23425-6\_13 : Efficient Hardware/Software Co-design for NTRU}.
\subsubsection{MOD-Q}
MOD-Q est un module implémentant l'algorithme de Barrett servant à effectuer de la réduction de modulos en temps constant. 

\subsection{Jeu d'instructions}
Le processeur utilisé est une déclinaison du RISC-V nommée RISCY. Il s'agit d'un processeur 32 bits avec un pipeline a 4 étages (IF, ID, EX et WB) implémentant les sets d'instructions de base
I (opérations sur entiers), C (instructions compressées) et M (multiplication et division d'entiers).\\
Le pipeline de base a été modifié afin d'y intégrer l'unité de calculs qui englobe les accélérateurs développés : dans l'étage EX, en plus de l'ALU (Arithmetic and Logic Unit) on retrouve une 
PQ-ALU (Post-Quantum ALU). \\ 
Afin d'utiliser cette dernière, quatre nouvelles instructions ont été ajoutées, toutes utilisant le format d'instructions R (elles ne necessitent pas l'utilisation d'immédiats). Un nouvel opcode 
est utilisé (bits 6 à 0 : 0x77 soit 0111 0111 en binaire) pour ces instructions. Le champ func3 (bits 14 à 12) est utilisé pour déterminer quel accélérateur doit être utilisé. Les champs rs1 (bits 
19 à 15), rs2 (bits 24 à 20) et rd (bits 11 à 7) sont utilisés en tant que buffers vers d'autres registres dans le GPR (General Purpose Register bank). Ce choix s'explique par le fait que les registres 
d'entrée (rs1 \& 2) et de sortie (rd) ne sont pas suffisement grands pour pouvoir contenir les opérandes et résultats des calculs effectués par la PQ-ALU. Ces instructions sont :\\
\begin{itemize}
    \item pq.mul\_ter
    \item pq.mul\_chien 
    \item pq.sha256 
    \item pq.modq
\end{itemize}
\newpage
\part{Analyse Critique}

\section{Choix du RISC-V}

On pourrait tout d'abord questionner le choix du RISC-V comme jeu d'instructions. Il existe en effet d'autres jeux d'instructions qui seraient en mesure d'implémenter ce genre d'algorithme.\\
x86 ou Arm viennent a l'esprit. 
\subsection{Arm}
Arm est très établis dans le monde des microcontroleurs notament avec son Cortex M4 ou le Cortex A53. Il est donc très courant que les cartes de développement soient équipées 
de processeurs Arm. Les résultats obtenus par une implémentation d'une solution à un problème donné sont ainsi très souvent donnés pour ce type de plateforme.\\
Cependant, le développement d'IP matérielles sur palteformes Arm n'est pas aisé particulièrement en raison de la nature fermée des technologies proposées. Il est donc difficile d'obtenir 
des accélérateurs hautement intégrés aux processeurs. Les durées de développement en sont également plus longues\\
Un autre désaventage majeur d'Arm est que l'utilisation et la modification de sa propriété intellectuelle est payante, ce qui rend le cout de développement de nouveau matériel plus élevé.\\
Enfin, on pourrait mentionner les potentiels conflits d'intérêts induits par le fait d'utiliser du matériel appartenant à une entreprise pour des applications de sécurité. En effet, dans 
un contexte géopolitique tendu, il serait légitime de se méfier de matériel produit par une entreprise basée dans un autre pays. 

\subsection{RISC-V}
RISC-V est un jeu d'instruction libre et open source. Par essence, toute personne est donc libre de le modifier a sa guise. Cette philisophie facilite grandement la facilité et la rapidité
de développement de nouveau matériel, notament grâce au partage de ressources au sein de la communauté.\\
De plus, il s'agit d'une ISA (Intrustion Set Architecture) modulaire, perettant aux développeurs d'y ajouter leur propres instructions. Cette propriété rend le RISC-V idéal pour la création 
d'accélérateurs matériels. Elle permet également de n'implémenter que ce qui est nécessaire au microcontroleur. Cela signifie qu'il est possible de minimiser la taille du circuit en évitant 
d'embarquer du matériel ne servant pas. C'est notament un avantage dans le cadre du développement de matériel embarqué dont l'espace est limité et dont la consommation énergétique doit être contenue. \\
Il est également possible d'en modifier directement le pipeline, ce qui permet d'y intégrer des accélérateurs matériels. Ce fort niveau d'intégration permet d'éviter les pertes de performances 
dûes à l'attente du processeur lors de l'utilisation des bus de communication.\\ 
En revenche, comme il a pû l'être constaté dans l'article proposé, il existe des disparités de performances entre les deux jeux d'instructions. Dans ce cas, de l'ordre de 20\%.


\section{SHA256 et Keccak}
L'article mentionne la possibilité de remplacer le module de génération polynomiale SHA256 par un module implémentant l'algorithme de Keccak. Ce 
changement n'a pas été effectué car :
\begin{itemize}
    \item Keccak a un chemin de données plus profond que SHA256 (1600 bits contre seulement 256 bits), ce qui signifie une plus grande complexité.
    \item SHA256 requiert beaucoup moins de ressources matérielles que Keccak, ce qui le rend potentiellement plus adapté à des applications embarquées. 
On remarque en effet dans les chiffres donnés une multiplication par 10 de l'utilisation de LUTs en utilisant Keccak (10.435 contre 1.031), et l'utilisation 
de 2,7 fois plus de registres.
\end{itemize} 
Le principal avantage de l'algorithme de Keccak serait ainsi une performance supérieure pour la génération de nombres polynomiaux en 
comparaison à SHA256. Afin de faire un choix informé sur le choix de l'algorithme le plus adapté, il serait intéressant remettre quelques 
aspects en perspective :
\begin{itemize}
    \item La carte de développement utilisée ici (Zynq Ultrascale+ ZCU102) est équipée de plusieurs centaines des milliers de LUTs \footnote{Voir site de Xilinx : \url{https://www.xilinx.com/products/boards-and-kits/ek-u1-zcu102-g.html} : "System Logic Cells (K) : 600"}. 
Une fois ces deux nombres côte à côte, l'augmentation engendrée par l'utilisation de l'algorithme de Keccak ne semble pas nécessairement conséquente (bien qu'elle 
ferait passer le nombre total de LUTs par le coeur RISC-V de 53.819 à 63.223 soit une augmentation de 20\% environ).

TODO : ucontroller
       aspect energie
       plus performant de combien ? Worth la perte de 20\% de ressources et energie ?
\end{itemize}


\section{Résultats}
La table II de l'article présente un profilage des temps d'exécutions de l'algorithme LAC. Elle y compare l'exécution d'un code de référence, c'est à dire non optimisé sur les plateformes
 ARM (Cortex M4) et RISC-V. On peut y voir que si la plateforme Arm est plus rapide à exécuter l'algorithme de référence que la plateforme RISC-V, cette dernière récupère un très net avantage 
une fois optimisée. Il aurait été intéressant de comparer ces résultats post-optimisation à une implémentation utilisant une plateforme Arm ayant été optimisée.\\
Enfin, les auteurs ne mentionnent pas la taille du code généré (bien que l'on puisse supposer que le RISC-V soit plus compact a ce niveau, étant donné qu'il a permi la création d'instructions 
dédiées aux quatres accélérateurs) ni l'emprunte mémoire de leur implémentation.
\end{document}
